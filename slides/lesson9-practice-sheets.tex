\documentclass[12pt]{article}
\usepackage[a4paper,margin=0.4in]{geometry}
\usepackage{xeCJK}
\usepackage{tikz}
\usepackage{tabularx}
\usepackage{calc}

\setCJKmainfont{Noto Sans CJK SC}

\pagestyle{empty}
\setlength{\parindent}{0pt}

% Base character box size
\newcommand{\charsize}{1.3cm}

% Draw a practice row: prompt box + empty boxes for writing
% #1 = prompt (pinyin or meaning), #2 = number of characters
\newcommand{\practicerow}[2]{%
  \begin{tikzpicture}[baseline=(current bounding box.center)]
    % Prompt box (wider, contains text)
    \node[draw, minimum height=\charsize, minimum width=2.5cm, align=center, font=\scriptsize] at (0,0) {#1};
    % Character boxes
    \foreach \i in {1,...,#2} {%
      \pgfmathsetmacro{\xpos}{2.5 + (\i-0.5)*\charsize}
      \draw (\xpos - \charsize/2, -\charsize/2) rectangle (\xpos + \charsize/2, \charsize/2);
      \draw[gray!30] (\xpos, -\charsize/2) -- (\xpos, \charsize/2);
      \draw[gray!30] (\xpos - \charsize/2, 0) -- (\xpos + \charsize/2, 0);
    }
  \end{tikzpicture}%
}

% Word block: multiple rows for one word
\newcommand{\wordblock}[2]{%
  \practicerow{#1}{#2}\\[3pt]
  \practicerow{}{#2}\\[3pt]
  \practicerow{}{#2}\\[3pt]
  \practicerow{}{#2}\\[3pt]
  \practicerow{}{#2}\\[0.25cm]
}

% Reference box at top
\newcommand{\refbox}[9]{%
  \begin{tabularx}{\textwidth}{|>{\centering\arraybackslash}X|>{\centering\arraybackslash}X|>{\centering\arraybackslash}X|}
    \hline
    \textbf{\Large #1} & \textbf{\Large #4} & \textbf{\Large #7} \\
    {\small #2} & {\small #5} & {\small #8} \\
    {\scriptsize #3} & {\scriptsize #6} & {\scriptsize #9} \\
    \hline
  \end{tabularx}
  \vspace{0.3cm}
}

\begin{document}

% Page 1
\refbox{进步}{jìnbù}{make progress}{实验室}{shíyànshì}{laboratory}{慢}{màn}{slow}
\wordblock{jìnbù}{2}
\wordblock{laboratory}{3}
\wordblock{slow}{1}

\newpage

% Page 2
\refbox{机会}{jīhuì}{opportunity}{语法}{yǔfǎ}{grammar}{重}{zhòng}{heavy}
\wordblock{opportunity}{2}
\wordblock{yǔfǎ}{2}
\wordblock{heavy}{1}

\newpage

% Page 3
\refbox{训练班}{xùnliànbān}{training class}{兴趣}{xìngqù}{interest}{能}{néng}{can}
\wordblock{training class}{3}
\wordblock{xìngqù}{2}
\wordblock{néng}{1}

\newpage

% Page 4
\refbox{中文桌子}{zhōngwén zhuōzi}{Chinese table}{平常}{píngcháng}{ordinary}{祝}{zhù}{wish}
\wordblock{Chinese table}{4}
\wordblock{ordinary}{2}
\wordblock{zhù}{1}

\newpage

% Page 5
\refbox{趁机会}{chènjīhuì}{take opportunity}{非常}{fēicháng}{extremely}{月}{yuè}{month}
\wordblock{chènjīhuì}{3}
\wordblock{extremely}{2}
\wordblock{yuè}{1}

\newpage

% Page 6
\refbox{只有…才}{zhǐyǒu...cái}{only if...then}{练习}{liànxí}{practice}{日}{rì}{day}
\wordblock{only if...then}{3}
\wordblock{liànxí}{2}
\wordblock{rì}{1}

\newpage

% Page 7
\refbox{敬祝}{jìngzhù}{respectful wish}{晚上}{wǎnshang}{evening}{今年}{jīnnián}{this year}
\wordblock{jìngzhù}{2}
\wordblock{evening}{2}
\wordblock{this year}{2}

\newpage

% Page 8
\refbox{敬上}{jìngshàng}{respectfully}{谈谈}{tántan}{chat}{结构}{jiégòu}{structure}
\wordblock{respectfully}{2}
\wordblock{tántan}{2}
\wordblock{structure}{2}

\newpage

% Page 9
\refbox{口语}{kǒuyǔ}{spoken language}{北京}{Běijīng}{Beijing}{设备}{shèbèi}{equipment}
\wordblock{spoken language}{2}
\wordblock{Běijīng}{2}
\wordblock{shèbèi}{2}

\newpage

% Page 10
\refbox{社会}{shèhuì}{society}{暑假}{shǔjià}{summer vacation}{希望}{xīwàng}{hope}
\wordblock{shèhuì}{2}
\wordblock{summer vacation}{2}
\wordblock{xīwàng}{2}

\newpage

% Page 11
\refbox{利用}{lìyòng}{make use of}{参加}{cānjiā}{participate}{计划}{jìhuà}{plan}
\wordblock{make use of}{2}
\wordblock{cānjiā}{2}
\wordblock{plan}{2}

\newpage

% Page 12
\refbox{支持}{zhīchí}{support}{短期}{duǎnqī}{short term}{健康}{jiànkāng}{health}
\wordblock{zhīchí}{2}
\wordblock{short term}{2}
\wordblock{jiànkāng}{2}

\newpage

% Page 13
\refbox{语言}{yǔyán}{language}{生活}{shēnghuó}{life}{女儿}{nǚ'ér}{daughter}
\wordblock{language}{2}
\wordblock{shēnghuó}{2}
\wordblock{nǚ'ér}{2}

\newpage

% Page 14 (last 2 words)
\begin{tabularx}{\textwidth}{|>{\centering\arraybackslash}X|>{\centering\arraybackslash}X|}
  \hline
  \textbf{\Large 随便} & \textbf{\Large 了解} \\
  {\small suíbiàn} & {\small liǎojiě} \\
  {\scriptsize casually} & {\scriptsize understand} \\
  \hline
\end{tabularx}
\vspace{0.3cm}

\wordblock{suíbiàn}{2}
\wordblock{understand}{2}

\end{document}
